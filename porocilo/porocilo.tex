\documentclass[11pt,a4paper]{article}
\usepackage[slovene]{babel}
\usepackage[utf8x]{inputenc}
\usepackage{graphicx}
\usepackage{hyperref}
\usepackage{pdfpages}
\usepackage{breakurl}
\usepackage{float}

\pagestyle{plain}

\begin{document}
\title{Poročilo pri predmetu \\
Analiza podatkov s programom R}
\author{Tatijana Slijepčević}
\maketitle

\pagebreak
\section{Izbira teme}
Naslov mojega projekta se glasi Lestvica 100 najboljših filmov vseh časov. Uporabila sem lestvico filmov, ki jo je naredil uporabnik spletne strani IMDb (povezava: 
\url http://www.imdb.com/list/ls055592025/), saj je poleg ocene filmov upošteval še njihovo priljubljenost ter število nominacij in nagrad.

Filme sem med seboj primerjala po:
\begin{itemize}
\item zvrsti,
\item letu nastanka,
\item številu nominacij,
\item številu nagrad,
\item oceni filma, 
\item državah, kje je bil posamezen film v večini posnet
\end{itemize}

\null
Zajela bom nominacije, ki so jih podelile organizacije oz. akademije The Oscars, The British Academy of Film and Television Arts (BAFTA) in The Golden Globe Award, ker so le-te najprestižnejše in najstarejše.  

\null
\null
Cilj projekta je ugotovitev:
\begin{itemize}
\item povprečne ocene filmov,
\item v katerem desetljetju je nastalo največ filmov z lestvice,
\item kateri žanr prevladuje,
\item koliko filmov je prejelo vsaj eno izmed nagrad.
\end{itemize}

\pagebreak
\section{Obdelava, uvoz in čiščenje podatkov}
Tabelo s podatki sem ročno naredila, ker druge možnosti ni bilo. Le-to sem kasneje uvozila v csv obliki. Po uvozu sem naredila 6 grafov, 4 stolpične diagrame in 2 tortna diagrama, filme pa sem preko teh primarjala na naslednji način:
\begin{itemize}
\item v prvem stolpičnem in drugem tortnem sem filme primerjala na podlagi držav, kjer so bili posneti,
\item v drugem stolpičnem po žanru,
\item v tretjem stolpičnem po oceni filmov,
\item v četrtem stolpičnem po letnicah nastanka,
\item v prvem tortnem pa koliko filmov je prejelo vsaj eno nagrado.
\end{itemize}

\null
\null
S pomočjo obdelave podatkov sem ugotovila naslednje:
\begin{itemize}
\item največ filmov je bilo posnetih v ZDA,
\item med žanri prevladuje drama,
\item med filmi je največ takšnih, ki imajo oceno med 8.0 in 8.5,
\item največ filmov je nastalo med letoma 1950 in 1960,
\item malo več kot desetina filmov ni prejela nobene nagrade.
\end{itemize}

 
\includepdf{../slike/stolpicni1.pdf}
\includepdf{../slike/stolpicni2.pdf}
\includepdf{../slike/stolpicni3.pdf}
\includepdf{../slike/stolpicni4.pdf}
\includepdf{../slike/tortni1.pdf}
\includepdf{../slike/tortni2.pdf}

\section{Analiza in vizualizacija podatkov}

V tretji fazi sem naredila zemljevid sveta, na katerem so prikazana števila filmov posnetih v posamezni državi.

\begin{figure}[H]
\includepdf[scale=0.6]{../slike/zemljevid.pdf}
\end{figure}

\pagebreak
\section{Napredna analiza podatkov}
 
V četrti fazi sem tako kot predhodno v Excelu ročno naredila tabele sto najboljših filmov v letih 2010, 2011, 2012, 2013 in 2014 (za vsako leto posebaj) in jih nato uvozila v csv obliki pod imeni top100-10, top100-11, top100-12, top100-13 in top100-14. Najprej sem za vsako tabelo ločeno naredila stolpični diagram, ki prikazuje distribucijo žanrov; datoteke z imeni: stolpicni10, stolpicni11, stolpicni12, stolpicni13, stolpicni14. Nato sem za vsako leto naredila zemljevid, ki prikazuje število filmov posnetih v posamezni državi; datoteke z imeni: zemljevid10, zemljevid11, zemljevid12, zemljevid13, zemljevid14.
Stari tabeli podatkov (filmi.csv) sem dodala stolpca BUDGET in BOX OFFICE. Nato sem za vsako tabelo posebaj izračunala minimalni in maksimalni propračun in zaslužek. Minimalnim zaslužkom, maksimalnim zaslužkom, minimalnim proračunom in maksimalnim proračunom sem priredila stolpične diagrame po letnicah; datoteke z imeni: min-box-office, max-box-officem, min-budget in max-budget. Izračunala sem še razmerje med maksimalnim in minimalnim (max/min) proračunom in zaslužkom po letnicah ter priredila stolpična diagrama pod imeni razmerje-boxoffice in razmerje-budget.

\includepdf{../slike/stolpicni10.pdf}
\includepdf{../slike/stolpicni11.pdf}
\includepdf{../slike/stolpicni12.pdf}
\includepdf{../slike/stolpicni13.pdf}
\includepdf{../slike/stolpicni14.pdf}

\includepdf{../slike/min-box-office.pdf}
\includepdf{../slike/max-box-office.pdf}
\includepdf{../slike/min-budget.pdf}
\includepdf{../slike/max-budget.pdf}

\includepdf{../slike/razmerje-boxoffice.pdf}
\includepdf{../slike/razmerje-budget.pdf}

\end{document}


