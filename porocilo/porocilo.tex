\documentclass[11pt,a4paper]{article}

\usepackage[slovene]{babel}
\usepackage[utf8x]{inputenc}
\usepackage{graphicx}

\pagestyle{plain}

\begin{document}
\title{Poročilo pri predmetu \\
Analiza podatkov s programom R}
\author{Študent FMF}
\maketitle

\section{Izbira teme}
Naslov mojega projekta se glasi Lestvica 100 najboljših filmov vseh časov. Uporabila bom lestvico filmov, ki jo je naredil uporabnik spletne strani IMDb (povezava: http://www.imdb.com/list/ls055592025/), saj je poleg ocene filmov upošteval še njihovo priljubljenost ter število nominacij in nagrad.

Filme bom med seboj primerjala po:
- zvrsti,
- letu nastanka,
- številu nominacij,
- številu nagrad,
- oceni filma, 
- državah, kje je bil posamezen film v večini posnet.

Zajela bom nominacije, ki so jih podelile organizacije oz. akademije The Oscars, The British Academy of Film and Television Arts (BAFTA) in The Golden Globe Award, ker so le-te najprestižnejše in najstarejše.  

Cilj projekta je ugotovitev:
- povprečne ocene filmov,
- v katerem desetljetju je nastalo največ filmov z lestvice,
- kateri žanr prevladuje,
- koliko filmov je prejelo eno izmed nagrad. 

\section{Obdelava, uvoz in čiščenje podatkov}

\section{Analiza in vizualizacija podatkov}

\includegraphics{../slike/povprecna_druzina.pdf}

\section{Napredna analiza podatkov}

\includegraphics{../slike/naselja.pdf}

\end{document}
